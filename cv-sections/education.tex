% !TEX root=../resume_cv.tex
%----------------------------------------------------------------------------------------
% SECTION TITLE
%----------------------------------------------------------------------------------------

\localexecute{en}{\cvsection{Education}}
\localexecute{it}{\cvsection{Istruzione}}


%----------------------------------------------------------------------------------------
% SECTION CONTENT
%----------------------------------------------------------------------------------------

\begin{cventries}
  %------------------------------------------------
                                                      
  \cventry{ % Degree
    \lang{en}{Master's Degree in Computer Engineering and Automation}
    \lang{it}{Laurea Magistrale in Ingegneria Informatica e dell'Automazione}
  } 
  {Università Politecnica delle Marche} % Institution
  { % Location
    \lang{en}{Ancona, Italy}
    \lang{it}{Ancona}
    }{ % Date
    \lang{en}{Sep. 2013 - Nowadays}
    \lang{it}{Set. 2013 - Presente}
    }{ % CVItems
    \begin{cvitems}
      % \item {
      %   \lang{en}{{\color{red}Info piano}}
      %   \lang{it}{Curriculum personalizzato orientato alla \emph{specializzazione in ingegneria informatica}.}
      % }
      \item {
        \lang{en}{{\color{red}Info competenze}}
        \lang{it}{Capacità e competenze rilevanti acquisite durante il corso di studi: 
          metodologie di progettazione \textit{Unified Process}
          $\cdot$ progettazione e sviluppo di \textit{sistemi software distribuiti} con particolare attenzione verso le \emph{piattaforme di Cloud Computing} e sui \textit{servizi RESTful} 
          $\cdot$ progettazione e sviluppo di \textit{applicazioni web} mediante l'utilizzo di  \textit{framework MVC} 
          $\cdot$ studio teorico delle tecniche di sicurezza in ambito informatico
          $\cdot$ Studio teorico e pratico delle \emph{Reti neurali} e degli algoritmi di classificazione nell'ambito dei \textit{Big Data}
          $\cdot$ studio teorico e pratico dei paradigmi di \textit{programmazione dichiarativa}, \textit{funzionale}, \textit{relazionale}, \textit{vincolata}, \textit{logica}, \textit{ad agenti}.
        }
      }
    \end{cvitems}
  }
                                              
  %--------------------------------------------------
                                            
  \cventry{
    \lang{en}{Bachelor's Degree in Computer Engineering and Automation}
    \lang{it}{Laurea Triennale in Ingegneria Informatica e dell'Automazione}
  } % Degree
  {Università Politecnica delle Marche} % Institution
  { % Location
    \lang{en}{Ancona, Italy}
    \lang{it}{Ancona}
    }{ % Date
    \lang{en}{Sep. 2009 - Feb. 2014}
    \lang{it}{Set. 2009 - Feb. 2014}
    }{ % Description(s) bullet points   
    \begin{cvitems}
      \item {
        \lang{en}{{\color{red}Info competenze}}
        \lang{it}{Capacità e competenze rilevanti acquisite durante il corso di studi: 
          tecniche di \textit{ingegneria del software} 
          $\cdot$ progettazione, sviluppo e mantenimento di \textit{basi di dati} 
          $\cdot$ studio dei principali problemi di \textit{ricerca operativa} e delle relative \emph{metodologie risolutive}
          $\cdot$ studio teorico delle reti di telecomunicazioni
          $\cdot$ studio teorico e pratico dei paradigmi di \textit{programmazione procedurale} e di \textit{programmazione ad oggetti}.
        }
      }
      \item {
        \lang{en}{Dissertation title: \emph{Implementative issue of an Android Sip Client for home automation applications}}
        \lang{it}{Tesi: \emph{Problematiche implementative di un Client Sip in ambienteAndroid per applicazioni domotiche}}
      }
      \item {
        \lang{en}{Final Mark: 97/110}
        \lang{it}{Voto Finale: 97/110}
      }
    \end{cvitems}
  }
  %------------------------------------------------
                                                          
\end{cventries}