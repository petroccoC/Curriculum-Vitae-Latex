% !TEX root=../resume_cv.tex

\localexecute{en}{\cvsection{Awards}}
\localexecute{it}{\cvsection{Premi}}

\begin{cvhonors}
																								
	%----------------------------------------------------------------------
																								
	\cvhonor{
		\lang{en}{3rd Place, \emph{HackUNIVPM}}
		\lang{it}{3° Posto, \emph{HackUNIVPM}}
		}{
		\lang{en}{
			This initiative is a 36 hours no-stop programming marathon which is open to all the students regularly enrolled at the Università Politecnica delle Marche. 
			It has been organized by the Information Engineering Department and the Management Department (respectively DII and DM) at the faculty of engineering.
			After this time every team of students had to deliver an innovative product concerning a main theme: in the 3rd edition it had been about \emph{exploiting innovative IoT solutions}.
			It had to be delivered a business plan along with a proof of work (a web application).
			Projects' score had been assigned according to how much the project was innovative and economically viable.
			We, \emph{BrightCode team}, delivered \emph{Connecter} by which we got the 3rd place.
		}
		%
		\lang{it}{
			Maratona di programmazione di 36 ore organizzata dai Dipartimenti di Ingegneria dell'Informazione e di Management dell'UNIVPM.
			Si deve consegnare la proof of work (un servizio web) e relativo business plan di un prodotto tenendo conto di un tema principale: questa terza edizione era votata alla ricerca di soluzioni \emph{IoT} per \emph{smart house}, \emph{smart city} ed \emph{industry 4.0}.
			Noi, gruppo \emph{BrightCode}, abbiamo consegnato \emph{Connecter}.
		}
		}{
		\lang{en}{Ancona, Italy}
		\lang{it}{Ancona}
	}
	{2016}	
							
																														
	%----------------------------------------------------------------------
																													
	\cvhonor{
		\lang{en}{3rd Place, \emph{VarGroup Innovation}}
		\lang{it}{3° Posto, \emph{VarGroup Innovation}}
		}{
		\lang{en}{{\color{red}Fill}}
		\lang{it}{
			Seconda edizione dell'hackathon organizzato da VarGroup.
			I gruppi partecipanti, messi in contatto con dei committenti, devono presentare una soluzione innovativa al problema che viene loro sottoposto.
			Noi, gruppo \emph{//noComment}, abbiamo sviluppato \emph{DressLab}, un sistema distribuito per boutique che sfrutta i chip rfid instalati nelle etichette dei prodotti per semplificarne la ricerca, l'inventariato e l'analisi delle vendite.
			Inoltre abbiamo presentato il prototipo di un \emph{camerino smart}, il quale sfrutta i tag rfid per individuare i capi al suo interno e suggerirne altri al cliente.
		}
		}{
		\lang{en}{Cervia (RA), Italy}
		\lang{it}{Cervia (RA)}
	}
	{2016}
																																							
	%----------------------------------------------------------------------
																																									
	\cvhonor{
		\lang{en}{1st Place, \emph{HackUNIVPM}}
		\lang{it}{1° Posto, \emph{HackUNIVPM}}
		}{
		\lang{en}{It had been the first edition of the HackUNIVPM initiave (refer to the above description for further details).
			The main theme of this edition had been about \emph{local tourism} and the new ways to promote it.
			We, \emph{BrightCode team}, delivered \emph{Triplovin} by which we got the 1st place.
			Triplovin is web application in which the locals - the so called \emph{lovers} - offer theirself as guides to the \emph{tripper}, by exploiting their knowledge about the territory.}
		\lang{it}{
			Prima edizione di \emph{HackUNIVPM}, focalizzata sulla promozione e lo sviluppo del \emph{turismo locale}.
			Noi, gruppo \emph{BrightCode}, abbiamo consegnato \emph{Triplovin}.
			Triplovin è una piattaforma web che offre un nuovo modello di fare e promuovere il turismo: gli abitanti locali - i così detti \emph{lovers} - si offrono di fare da guida, mettendo a disposizione la loro profonda conoscenza del territorio, ai turisti (\emph{tripper}).
		}
		}{
		\lang{en}{Ancona, Italy}
		\lang{it}{Ancona}
	}
	{2014}{last}
																																																
	%----------------------------------------------------------------------
																																																
\end{cvhonors}